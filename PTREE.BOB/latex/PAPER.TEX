\documentstyle [12pt]{article}
\nofiles                           % no aux files

% get 0.5in top, left and right margins, and increase page height
\oddsidemargin=0in  \evensidemargin=0in  \textwidth=6.3in
\topmargin=-1in                         \textheight=9.5in

\begin{document}

\title{Genealogical-tree probabilities in the infinitely-many-site model}

\author{R.C. Griffiths\\
Mathematics Department \\
Monash University\\
Clayton,Vic. 3168\\
Australia\\}

\maketitle
\begin{abstract}
    This paper considers the distribution of the genealogical tree of a 
sample of genes in the infinitely-many-site model where the relative
age ordering of the mutations (nodes in the tree) is known.
A computer implementation of a recursion for the probability of such trees
is discussed when the nodes are age-labeled, or not. \\
\vspace{0.3cm}
{\bf Key words} : Genealogical Trees, Infinitely-many-site model, Population
Genetics
\end{abstract}
\section{Introduction}    

    Ethier and Griffiths (1987) study a reformulation of Watterson's (1975)
infinitely-many-site model in population genetics as a measure-valued
diffusion process. A gene is thought of as an infinitely long sequence
of completely linked sites where mutations occur at sites never
before mutated. Sites within a gene are labeled by elements
of [0,1] and genes have a type space $E = [0,1]^{Z_+}$.
A gene is of type ${\bf  x} = (x_0,x_1,\ldots) ~\in~ E$ if $x_0,x_1,\ldots$
is the time-ordered sequence of sites 
at which mutations have occurred in the line of 
descent of that gene, where $x_0$ is the most recently mutated site.
If the process is considered under a stationary 
distribution then a collection of $n$ genes always forms a genealogical 
tree as defined below. 

Formally let $n ~\in~ N$. Then $({\bf x}_1,\ldots ,{\bf x}_n) ~\in~ E^n$,
is a {\em  tree }if : 

~~~~~each sequence ${\bf x}_i$ has distinct coordinates for fixed
$i ~\in~ \{1, \ldots ,n \}$;\hfill (1.1)

~~~~~if $i,i' ~\in~ \{1, \ldots ,n\},~j,j' ~\in~ Z_+$ and $x_{i,j}
= x_{i',j'},$ 
 then $x_{i,j+k} = x_{i',j'+k}$

~~~~~for $k = 0,1, \ldots ;$\hfill (1.2)

~~~~~there exist $j_1, \ldots ,j_n ~\in~ Z_+$ such that
$x_{1,j_1} = \cdots = x_{n,j_n}. $\hfill (1.3)

Given $n~\in~ N$, let ${\cal T}_n = \{({\bf x}_1, \ldots ,{\bf x}_n)
~{\rm is~a~tree} \}.$
Define equivalence relations $\sim$ and $\approx$ 
by $({\bf x}_1, \ldots ,{\bf x}_n)
\sim ({\bf y}_1,\ldots ,{\bf y}_n)$ if $\exists$ a bijection 
$\zeta : [0,1] \mapsto [0,1]$
with $y_{i,j} = \zeta (x_{i,j})$ for $i = 1, \ldots ,n$ and $j = 0,1, \ldots$
and by $({\bf x}_1, \ldots ,{\bf x}_n) \approx ({\bf y}_1, \ldots ,{\bf y}_n)$
if $\exists$ a bijection $ \zeta : \mapsto [0,1]$ and a permutation $\sigma$ of
$(1, \ldots ,n)$ such that $y_{ \sigma (i),j} = \zeta(x_{i,j})$ for 
$i = 1, \ldots ,n$ and $j = 0,1, \ldots .$ Equivalence classes in 
${\cal T}_n /\!\sim$ are related to labeled trees and in 
${\cal T}_n /\!\approx$ to unlabeled trees. Griffiths (1987) uses
this relationship to count genealogical trees. Denote
$$({\cal T}_d /\!\sim)_{\circ} = \{T ~\in~ {\cal T}_d /\!\sim~:~
{\bf x}_1, \ldots ,{\bf x}_d{\rm ~are~distinct~for~all~}
({\bf x}_1, \ldots, {\bf x}_d)~\in
~T \}$$ and similarly for $({\cal T}_d /\!\approx )_{\circ},$
where we do not distinguish between an equivalence class and a typical 
member.

Let $T~\in~\bigcup_d ({\cal T}_d /\!\sim)_{\circ}$, then $T$ can be thought 
of as a labeled graph-theoretic tree. Taking the {\em root} of the tree 
as the first common coordinate of representative sequences 
$({\bf x}_1, \ldots ,{\bf x}_d)~\in~T$, then $T$ is a rooted labeled tree
whose root has at least two edges attached.
The {\em nodes} of the tree are the distinct coordinates of
$({\bf x}_1, \ldots, {\bf x}_d)$, up to and including the root.
The {\em leaves} of the tree are the first coordinates of the sequences.
In $n$ possibly non-distinct sequences the graph-theoretic tree is thought
of as the tree constructed from distinct sequences among the $n$.
(This is different from the usage in Griffiths (1987) where distinct nodes 
are appended to the beginning of each sequence.)

    The basic site-type space [0,1] is not critical in the description
of a labeled tree. Indeed, later in this paper, nodes will be labeled by 
non-negative integers 
according to their relative ages as mutations back in time.

Let $p(T,{\bf  n})$ be the probability of obtaining the alleles
$T~\in~({\cal T}_d /\!\sim)_{\circ}$ at equilibrium with multiplicities
${\bf  n} = (n_1, \ldots ,n_d)$.
Ethier and Griffiths (1987) derive the recursion \hfill \\
$$n(n - 1 + \theta )p(T,{\bf  n}) =
\sum_{k:n_k \geq 2}n_k(n_k - 1)p(T,{\bf  n - e}_k)~~~~~~~~~~~~~~~~~~~~~~~~ $$
$$~~~~~+~~\theta \sum_{{k:n_k = 1,\atop x_{k,0}{\rm ~distinct},}\atop {\cal S}
{\bf x}_k \not= {\bf x}_j~\forall ~j}
p({\cal S}_kT,{\bf  n})$$
$$~~~~~~~~~~~~~~~~~~~+~\theta \sum_{k:n_k=1,\atop x_{k,0}{\rm ~distinct.}~}
\sum_{j:{\cal S}{\bf x}_k = {\bf x}_j}p({\cal R}_kT,{\cal R}_k({\bf n + e}_j)).
\eqno(1.4)$$

The boundary condition is $p(T_1,(1)) = 1,~T_1~\in~{\cal T}_1 /\!\sim.$
 ${\cal S}$ is a shift operator which deletes the first coordinate 
({\em i.e.}, the last mutation) of a sequence.
${\cal S}_kT$  deletes the first coordinate of the $k th$ 
sequence of $T$. ${\cal R}_kT$ removes the $k th$ sequence of $T$.
$\theta$ is a scaled mutation parameter.
The {\em degree} of a tree $(T,{\bf n})$ is defined as 
$n - 1$ + the number of nodes of $T$
and $(1.4)$ is recursive in this degree.

    Let $p^*(T,{\bf  n})$ be the probability of a corresponding unlabeled
tree in $({\cal T}_d /\!\approx )_{\circ}$ with multiplicity of the sequences
given by ${\bf  n}$. $p^*$ is related to $p$ by a combinatorial factor.
Let $P_d$ denote the set of permutations of $(1, \ldots ,d)$.
Given $T~\in~{\cal T}_d /\!\sim $ and $\sigma~\in~P_d,$ define
$T_{\sigma} = \{({\bf x}_{\sigma (1)}, \ldots ,{\bf x}_{\sigma (d)}) :
({\bf x}_1, \ldots ,{\bf x}_d)~\in~T\}$ and
${\bf  n_{\sigma}} = (n_{\sigma (1)}, \ldots ,n_{\sigma (d)}).$
Let $a(T,{\bf  n}) = |\{\sigma~\in~P_d : 
T_{\sigma } = T, {\bf  n}_{\sigma} = {\bf  n}\}|,$ then
$$ p^*(T,{\bf  n}) = \frac{n!}{n_1! \ldots n_d! a(T,{\bf  n})}p(T,{\bf  n}).$$


    In population genetics an important death process is the coalescent
process (Kingman (1982)). This describes the lines of
descent of a sample of n genes in the diffusion time scale. A review 
article is Tavar\'{e} (1984).

    Let $T_n, \ldots ,T_2$ be the times spent when there are $n, n - 1, \ldots
, 2$ ancestors of a sample of $n$ genes. These are mutually independent,
exponential random variables with rate parameters 
$\frac{1}{2}n(n - 1), \ldots ,\frac{1}{2}.2.1$ . Mutations occur in a
Poisson process of rate $\frac{1}{2} \theta$ along the edges of the
binary splitting tree formed by the coalescent process, conditional on
$T_n ,\ldots ,T_2$ . Label these mutations by mutually independent,
identically distributed uniform random variables in [0,1]. Progressing
backwards in time along the coalescent tree from the genes in the sample
to the common ancestor the sequences of mutations along the paths,
${\bf x}_1, \ldots ,{\bf x}_n$, form a tree satisfying (1.1) -- (1.3) which
has a probability distribution satisfying (1.4). The recursion (1.4)
is derived by using the generator in the measure-valued diffusion process.
The equivalence of this and the coalescent approach is shown in Theorem 5.6
of Ethier and Griffiths (1987).

    An alternative way of realizing the coalescent process with mutation
is to consider both the death process and the mutation process 
simultaneously. Then if there are $r$ ancestors of a sample, the 
waiting time until the next event backwards in time 
has an exponential distribution with rate parameter
$\frac{1}{2}r(r + \theta - 1)$. The event is a
mutation, with probability $\theta /(r + \theta - 1)$, or a coalescence,
with probability $(r - 1)/(r + \theta - 1)$.

    In this paper the stochastic mechanism generating trees will be
supposed to be the coalescent process, and results derived from this
framework.

    In a sample of $n$ genes the distribution of the configuration of the
first coordinates of the sequence labels is the Ewens' sampling distribution
(Ewens (1972)). 
Upon mutation a gene is given a first
coordinate label completely new to the population, 
 so marginally the first coordinates behave as labels of genes 
in the infinitely-many-alleles model. 
    Let $\alpha (i) , i = 1,\ldots ,n $ be the number of alleles in a sample
of $n$ genes with $i$ representative genes and $d = \sum \alpha (i)$.
Then the probability of this configuration is Ewens' sampling formula
$$\frac{n!\theta^{d - 1}}{\alpha (1)!\ldots \alpha (n)!
1^{\alpha (1)} \ldots n^{\alpha (n)}(1+\theta ) \cdots (n - 1 +\theta )}~.$$

    Donnelly and Tavar\'{e} (1987) prove that if the alleles in the
sample are age-ordered from oldest to youngest and $\eta_i$ is the number
of genes of the $ith$ oldest allele in the sample, $i = 1, \ldots, d$, 
then the probability of such a configuration is
$$\frac{n!\theta^{d-1}}{\eta_d (\eta_d + \eta_{d-1})\cdots (\eta_d + \ldots
+ \eta_1)(1 + \theta ) \cdots (n - 1 + \theta )}~.\eqno(1.5)$$

    The number of segregating sites in a tree $T$ is the number of 
nodes $-~1$ . Watterson (1975) derives the probability generating function
of the number of segregating sites in a sample of $n$ genes as
$$H_n(z) = \prod_{j=2}^n\frac{j-1}{j - 1 + \theta  - \theta z}.\eqno(1.6)$$

    This paper extends (1.1)--(1.3) to trees whose nodes are labeled
according to their relative ages, and gives an analogous recursion to (1.4).
The relative ages of mutations in a line of descent is determined by the
structure of $x~\in~E$ even if they occur together in the same coalescent 
branch. Thus age-ordering mainly fixes the relative ages between divergent
lines of descent.

    A method of simulating an age-labeled tree and the
relationship to a birth process with immigration is discussed.

    A computer implementation of (1.4) and the age version (2.1) is discussed.
Table 1 shows probabilities
of all trees with sample sizes 2,3 and 1,2,3 nodes.

    A sample can be partitioned into the ancestral allele types in its line of
descent from a common ancestor. This is explored in Theorem 3 and the
following material. A formula for the expected partition into these allele
types is (3.5). 
A limit as the sample size tends to infinity provides a partition 
of the population into
ancestral allele types. A tabulation 
of the expected partition of the population is shown in Table 2.
\section{Age-labeled trees}

    An age-labeled tree is constructed from the coalescent process
by labeling the mutations occurring on the edges of the binary ancestral
tree by consecutive integers representing relative age ordering of the
mutations. A convention will be that the root type has label $0$ , and the
youngest mutation backward in time has the largest label. 
In a sample size of $n$ form the
sequences ${\bf y}_1, \ldots ,{\bf y}_n$ of mutation labels from the leaves to 
the root. If there are $d$ distinct sequences ${\bf x}_1, \ldots ,{\bf x}_d$
among the $n$, arrange them by the magnitude of their first coordinates, 
so that ${\bf x}_1$ represents the oldest allele, and ${\bf x}_d$ the
youngest. The earlier tree notation will be abused slightly, by letting
$T$ denote a particular age-labeled tree, rather than an equivalence class
of sequences. The ages of the nodes of trees are not included in Ethier
and Griffiths' measure-valued diffusion. Ethier (1989) includes information
about ages in a diffusion process, though 
genealogical trees are not constructed in his process.
\vspace{0.5cm}

{\em  Theorem} 1. 
     Let $T$ be a tree represented by distinct age-labeled, ordered sequences
${\bf x}_1, \ldots ,{\bf x}_d$ of multiplicity ${\bf  n} = (n_1, \ldots ,n_d)$
in a sample of $n$ genes. The probability of obtaining a particular
ordered sample of these sequences $p_a(T,{\bf  n})$ satisfies the recursion
$$n(n - 1 + \theta)p_a(T,{\bf  n}) = \sum_{k:n_k \geq 2}n_k(n_k - 1)
p_a(T,{\bf n - e}_k) + \theta \delta_{n_d,1}p_a(T',{\bf  n'})~,\eqno(2.1)$$ 
where $T'$ is the tree formed by reordering the distinct sequences
of ${\bf x}_1, \ldots ,{\bf x}_{d-1},{\cal S}{\bf x}_d$ according to the age of
their first coordinates and ${\bf  n'}$ the corresponding multiplicities.
The boundary condition is $p_a(T_1,(1)) = 1.$
Recursion is on the degree of the trees.

    The probability of obtaining an unordered sample of age-labeled sequences 
${\bf x}_1, \ldots ,{\bf x}_d$ is
$$p^*_a(T,{\bf  n}) = \frac{n!}{n_1! \ldots n_d!}p_a(T,{\bf  n}).\eqno(2.2)$$
\vspace{0.5cm}

{\em  Proof.}
     Consider the probability of obtaining sequences 
${\bf x}_1, \ldots ,{\bf x}_d$
such that the first $n_1$ are of type ${\bf  x}_1, \ldots$ , and the last $n_d$
are of type ${\bf x}_d$.

    Conditional on the last event being a coalescence, the probability of
obtaining such a sample is
$$\sum_{k:n_k \geq 2}\frac{n_1! \ldots n_d!}{n!}
\frac{n_k - 1}{n - 1}
\frac{(n - 1)!}{n_1! \ldots (n_k - 1)! \ldots n_d!}
p_a(T,{\bf  n - e}_k).\eqno(2.3)$$
At the instant before coalescence the $n - 1$ genes are in an unordered 
arrangement, and the probability of an ordered arrangement is required.
The factor $(n_k - 1)/(n - 1)$ is the probability that the parent is
from the $n_k -1$ genes with label ${\bf x}_k$ before coalescence. 

    The last event can only be a mutation if $n_d = 1$. There are two cases
to consider conditional on mutation. If ${\cal S}{\bf x}_d$ is still a 
singleton sequence the conditional probability of obtaining the ordered
arrangement is
$$\frac{1}{n}p_a(T',{\bf  n'}),\eqno(2.4a)$$
since mutation must occur on a particular gene. 
If ${\cal S}{\bf x}_d = {\bf x}_k$
the probability is 
$$\frac{n_1! \ldots n_d!}{n!}
\frac{n_k + 1}{n}
\frac{n!}{n_1! \ldots (n_k + 1)! \ldots n_{d - 1}!}p_a(T',{\bf  n'}).
\eqno(2.4b)$$
Multiplying (2.2) by $(n - 1)/(n - 1 + \theta)$ and (2.4a) or (2.4b)
by $\theta /(n - 1 + \theta)$ gives (2.1).
\vspace{0.5cm}

    A simple case is of an age-ordered tree in a sample of 2, with paths
to the root 
        $${\bf x}_1 = (i_0,i_1, \ldots ,i_{k - 1},0),~~
        {\bf  x}_2 = (j_0,j_1, \ldots ,j_{\ell - 1},0).$$
The coordinates are disjoint, apart from 0, and decreasing within
sequences.\\
Let ${\bf  n} = (1,1)$; then
    $$p^*_a(T,{\bf  n}) = \left (\frac{1}{2}\right )^{k + \ell - 1}
\left (\frac{ \theta }{1 + \theta}\right )^{k + \ell}\frac{1}{1 + \theta}.$$


    The recursion (1.4) of Griffiths and Ethier (1987) has an alternative 
proof as a corollary of Theorem 1 (with sequences of integers instead
of elements of [0,1] ).
\vspace{0.5cm}

{\em  Corollary.}
  The recursion (1.4) for $p(T,{\bf  n})$ holds.
\vspace{0.5cm}

{\em  Proof.}
       Let the youngest node have a label $s$ and the sequences be of lengths
$m_1 , \ldots , m_d$. Denote by ${\cal T}$ the set of age-labeled trees
$$\{({\bf y}_1, \ldots , {\bf y}_d) : y_{i,j} = x_{i, \sigma (j)},
j = 1, \ldots ,m_i,~\sigma~\in~P_s,~y_{i,0} > y_{i,1} > ... > y_{i,m_i},
i = 1, \ldots ,d \}.$$
Let $Y$ be a tree constructed from 
$({\bf y}_1, \ldots ,{\bf y}_d)$ in ${\cal T}$
and $p'_a(Y,{\bf  n})$ its probability.
Then $p'_a(Y,{\bf  n})$ satisfies a similar equation to (2.1), but in the last
term on the right the youngest sequence is not necessarily ${\bf  y}_d$.
Then
    $$p(T,{\bf  n}) = \sum_{({\bf y}_1, \ldots ,{\bf y}_d)~\in~{\cal T}}
        p'_a(Y,{\bf  n})~.$$
Summation in the similar equation gives (1.4). Note that in the youngest
sequence ${\bf y}_k$, (say) $y_{k,0}$ is always a leaf node of the tree.
The terms in the last two summations on the right of (1.4) are formed 
when the singleton sequence ${\bf  y}_k$ is the youngest.
\vspace{0.5cm}

    All possible trees $T$ with $n = 2,3$ and the number of nodes $1,2,3,4$
 are enumerated and $p^*(T,{\bf n})$ is calculated from (1.4) 
for illustrative values
of $\theta$ in Table 1.
\newpage
\vspace{0.5cm}
\begin{Large}
\begin{center}
{\bf Table 1}
\end{center}
\end{Large}
\vspace{0.5cm}
\begin{large}
\begin{center}
{\bf Tree probabilities with sample sizes two and three.}
\end{center}
\end{large}
\begin{small}
\begin{tabular}{lllllll}
& & & &$\theta$ \\
Sequences,multiplicities 
&   0.1  &   0.5  &   1.0  &   1.5  &   2.0  &   2.5 \\ \cline{2-7} 
(0),2             &0.909091 &0.666667 &0.500000 &0.400000 &0.333333 &0.285714 \\
(0),1;(1,0),1     &0.082645 &0.222222 &0.250000 &0.240000 &0.222222 &0.204082 \\
(0),1;(2,1,0),1   &0.003757 &0.037037 &0.062500 &0.072000 &0.074074 &0.072886 \\
(1,0),1;(2,0),1 &0.003757 &0.037037 &0.062500 &0.072000 &0.074074 &0.072886 \\
(0),1;(3,2,1,0),1 &0.000171 &0.006173 &0.015625 &0.021600 &0.024691 &0.026031 \\
(1,0),1;(3,2,0),1 &0.000512 &0.018519 &0.046875 &0.064800 &0.074074 &0.078092 \\
(0),3             &0.865801 &0.533333 &0.333333 &0.228571 &0.166667 &0.126984 \\
(0),2;(1,0),1     &0.080583 &0.195556 &0.194444 &0.166531 &0.138889 &0.115898 \\
(0),1;(1,0),2     &0.039355 &0.088889 &0.083333 &0.068571 &0.055556 &0.045351 \\
(0),2;(2,1,0),1   &0.003068 &0.027852 &0.042438 &0.044362 &0.041667 &0.037660 \\
(0),1;(2,1,0),2   &0.001789 &0.014815 &0.020833 &0.020571 &0.018519 &0.016197 \\
(1,0),1;(2,0),2   &0.004202 &0.035556 &0.050926 &0.050939 &0.046296 &0.040792 \\
(0),1;(1,0),1;(2,0);1
                  &0.002558 &0.026074 &0.043210 &0.047580 &0.046296 &0.042925 \\
(0),1;(1,0),1;(2,1,0),1
                  &0.001249 &0.011852 &0.018519 &0.019592 &0.018519 &0.016797 \\
(0),2;(3,2,1,0),1 &0.000130 &0.004326 &0.009924 &0.012509 &0.013117 &0.012759 \\
(0),1;(3,2,1,0),2 &0.000081 &0.002469 &0.005208 &0.006171 &0.006173 &0.005785 \\
(1,0),1;(3,2,0),2 &0.000272 &0.008395 &0.017940 &0.021453 &0.021605 &0.020353 \\
(1,0),2;(3,2,0),1 &0.000311 &0.009778 &0.021283 &0.025791 &0.026235 &0.024908 \\
(1,0),1;(2,0),1;(3,0),1
                  &0.000041 &0.001738 &0.004801 &0.006797 &0.007716 &0.007949 \\
(0),1;(2,1,0),1;(3,0),1
                  &0.000179 &0.007190 &0.019033 &0.026269 &0.029321 &0.029846 \\
(0),1;(2,1,0),1;(3,2,1,0),1
                  &0.000057 &0.001975 &0.004630 &0.005878 &0.006173 &0.005999 \\
(1,0),1;(2,0),1;(3,2,0),1
                  &0.000153 &0.005531 &0.013374 &0.017353 &0.018519 &0.018219 \\(0),1;(1,0),1;(3,2,1,0),1
                  &0.000020 &0.000790 &0.002058 &0.002799 &0.003086 &0.003111 \\
(0),1;(3,1,0),1;(2,1,0),1
                  &0.000020 &0.000790 &0.002058 &0.002799 &0.003086 &0.003111
\end{tabular}
\end{small}
\vspace{0.5cm}

    As an example of an age-labeled tree $T$
consider two ordered sequences, $(y,z)$ of multiplicity
2 and $(w,x,z)$ of multiplicity 1.
The probabilities $p_a^*(T,{\bf  n})$ of all the different possible 
age-labeled sequences\\ 
$$(1,0),(3,2,0)~;~(2,0),(3,1,0)~;~(3,0),(2,1,0)$$
are respectively
$$\frac{\theta^3(19\theta^2+62\theta + 52)}{36(1+\theta )^4(2+\theta )^3}~; 
\frac{\theta^3(5\theta + 8)}{12(1+\theta )^4(2 + \theta )^2}~;
\frac{\theta^3}{4(1+\theta )^4(2 + \theta )}~.$$
If the age ordering of the nodes is unknown, adding the above probabilities,
$$p^*(T,{\bf n}) = \frac{\theta^3(43\theta^2 + 152\theta + 136)}
{36(1 + \theta )^4(2 + \theta )^3}~.$$

    It is possible to estimate $\theta$ numerically by maximum likelihood.
Respective estimates, standard deviations and probabilities 
$\left (\hat{\theta }, {\rm sd}(\hat{\theta }), 
p_a^*(T,{\bf n};\hat{\theta })\right )$ in this example
for the three labeled trees are (1.88,1.75,0.01083), (1.82,1.69,0.00931),
(1.73,1.59,0.00625) and for the combined estimate from $p^*(T,{\bf n})$, 
 (1.82,1.69,0.02637).

	In Table 1 most of the trees have a maximum probability with respect
to $\theta$ for $\theta ~\in ~(0.1,2.5)$.

\vspace{0.5cm}
    A tree-growing simulation from Ethier and Griffiths (1987) can be adapted
to age-labeled trees.

    Consider a discrete-time Markov process whose state space is collections of
unordered sequences of increasing integers representing age-labeled trees.\\
1. Begin at $\tau = 2$ with two identical sequences
$\{{\bf x}_1 = (0),~{\bf  x}_2 = (0)\}.$\\
2. Suppose at a particular time $\tau$ the state of the process is
$\{{\bf x}_1, \ldots ,{\bf x}_m\}$ and the largest first coordinate is $s$. 
At time
$\tau + 1$ make a transition by either\\
(i) duplicating one of the sequences
$$\{{\bf x}_1, \ldots ,{\bf x}_m\} \mapsto 
\{{\bf x}_1, \ldots ,{\bf x}_k,{\bf x}_k,{\bf x}_{k+1}, \ldots
,{\bf x}_m\}$$
with probability $(m - 1)/m(m + \theta -1)~~k = 1, \ldots , m$ ; or\\
(ii) adding a term to one of the sequences
$$\{{\bf x}_1, \ldots ,{\bf x}_j, \ldots ,{\bf x}_m\} \mapsto 
\{{\bf x}_1, \ldots , 
{\bf x}_{j-1},(s+1,{\bf x}_j),{\bf x}_{j+1}, \ldots ,{\bf x}_m\}$$
with probability $\theta/ m(m + \theta - 1)~,~j = 1, \ldots ,m$;\\
3. Stop when the number of sequences is $n+1$ for the first time and discard
the last duplicated sequence. The $n$ sequences left form an age-labeled tree.
At time $\tau$ the tree grown is of degree $\tau$.

    The proof that (1) -- (3) lead to the recursion (2.1) is similar to that in
Ethier and Griffiths (1987), p542, but is simpler, since $a(T,{\bf  n})$
there is replaced by unity. It is also possible to show that an event of the type 2(ii) in the simulation corresponds to an ancestral mutation while there are $m$ ancestors of the sample of $n$. 

    Hoppe (1984,1987) considers an urn scheme for simulating an age-ordered
sample in the infinitely-many-alleles model which has a similarity to
(1) -- (3).
\vspace{0.5cm}

    Tavar\'{e} (1987) relates a linear birth process
$\{Z(t),~t \geq 0,~Z(0) = 0\}$ with birth rate $\lambda_n = n$ ,
and immigration rate $\theta$ , to (1.5) . If $\{N_i(t), i = 1,2, \ldots \}$
is a partition of $Z(t)$ into family sizes of immigrants arriving at times
$0 < t_1 < t_2 < \ldots$ then
$$P(\{N_i(t) = \eta_i~;~i = 1, 2, \ldots , k\}|~Z(t) = \sum_1^k\eta_i)$$
is identical to (1.5).

    The next theorem uses Tavar\'{e}'s idea to give a similar
interpretation to the simulation (1) -- (3). Note, however, that the process
is fundamentally different from Tavar\'{e}'s. Immigrant alleles in the
process of Theorem 2 may be destroyed, and the birth rate 
when there are $n$ individuals is $\lambda_n = n - 1$
 , not $\lambda_n = n$ . Perhaps {\em immigrant} is a misnomer here.
\vspace{0.5cm}

    {\em Theorem} 2.  Let $\{Z(t),~t\geq 0,Z(0) = 1\}$ 
be the population size in a
continuous time Markov birth process with birth rates
$\lambda_1 = 1$ and $\lambda_n = n - 1~,~n = 2,\ldots$ . Immigrants arrive
into the population independently as a Poisson process of rate $\theta $
but (contrary to usual) do not increase the population size.

    Construct a tree from the process by assigning a root node $0$ to the last
immigrant before $Z(t) = 2$, then nodes $1,2,\ldots $ to successive
immigrants. A new node and edge 
is joined to one of the existing end edges of the tree
chosen at random. A new edge is appended to its parent node when a birth
occurs. A formal construction of sequences representing a tree is analogous
to $2(i)$ and $2(ii)$ corresponding to birth or immigration. Let 
$(T(t),{\bf n}(t))$ be the tree constructed at time $t$, containing only the
labeled nodes.
Denote the random time $t_n = {\rm sup} \{t~;Z(t) \leq n\}$. Then
$$p_a^*(T,{\bf n}) = P(T(t_n) = T,{\bf n}(t_n) = {\bf n})~.$$

    {\em Proof.} The proof is almost immediate by noting that the transitions
$2(i)$ and $2(ii)$ agree with the imbedded Markov chain from the continuous
process where births and immigration occur, and that step $3$ corresponds to
stopping the $n$ sequence tree at $t_n$ .
\section{Allele frequencies}

    Of interest is the number of genes in a sample of $n$ of the $jth$
oldest allele in the history of the sample.
Such an allele may, or may not, be represented in the sample.
Theorem 3 gives a recursion for the distribution of the number of 
genes of this type in a sample. 
A corollary considers the number of genes of the common ancestor's type in a
sample. Beder (1988) and Griffiths (1986) consider similar problems. 
\vspace{0.5cm}

    {\em Theorem} 3. Let $T$ be a tree represented by $n$ age-labeled
sequences and\\ 
$q_n(r,s;j)
               = P($\{$T$ has $s+1$ nodes and the allele of the $jth$ node 
has multiplicity $r$ in the sample.\})\\
Then \hfill \\
$n(n-1+\theta )q_n(r,s;j) = n(r - 1)q_{n-1}(r-1,s;j) 
~+~n(n - 1 - r)q_{n-1}(r,s;j)
\hfill $
$$~+~(n - r)\theta q_n(r,s-1;j)~+~
(r + 1)\theta q_n(r+1,s-1;j)~+~
\delta_{j,s} \delta_{r,1} n\theta Q_n(s - 1)~,\eqno(3.1)$$
$n = 2,3,\ldots~,~s = 0,1\ldots~,~j = 0,\ldots , s~,~r = 0,\ldots ,n$
and where $Q_n(s)$ is the probability that $T$ has $s+1$ nodes,
with probability generating function (1.6). The boundary condition
is $q_1(r,s;j) = 1$ if $j = s = 0~,r = 1$ and zero otherwise.
Interpret $q_n(r,s;j)$ as $0$ if $j > s$ or $r > n$ or $r,s < 0$.
A particular case is 
$$q_n(r,0;0) = \delta_{r,n}\prod_{j = 2}^n
\frac{j-1}{j - 1 + \theta }~.$$
\vspace{0.5cm}

    {\em Proof.} Consider the coalescent process and whether the last event
is a coalescence with probability $(n-1)/(n-1+\theta )$ or a mutation
with probability $\theta /(n-1+\theta )$.

    The respective terms in (3.1) are derived by considering whether :\hfill \\
(i) the coalescent parent has an allele type of the $jth$ node;\hfill \\
(ii) the coalescent parent does not have an allele type of the $jth$ node;
\hfill \\
(iii) the last mutation occurs on a gene with an allele type of the $jth$ node;
\hfill \\
(iv) the last mutation occurs on a gene with an allele type 
not of the $jth$ node.
\vspace{0.5cm}

    Let $\mu_n(s;j)$ be the probability that  
a sample of $n$ genes has a tree $T$ with $s + 1$ nodes 
and that a randomly chosen
gene is the same allele type as the $jth$ node.
Then \hfill \\
$$n(n - 1 + \theta )\mu_n(s;j) = n(n - 1)\mu_{n - 1}(s;j) 
~+~ \theta (n - 1) \mu_n(s - 1;j)~+ \theta \delta_{j,s}Q_n(s - 1)~,\eqno(3.2)$$
$n = 2,3, \ldots ~,s = 0,1,\ldots ~,j = 0,\ldots ,s$ . The boundary condition
is $\mu_1(s;j) = \delta_{j,0} \delta_{s,0}~.$ The expected proportion
of genes which are the same allele type as the $jth$ node, given the tree has
$s + 1$ nodes is $\mu_n(s;j)/Q_n(s)$ . This has a similar proof to
(3.1). A recursion from (1.6) (or argued directly) is
$$(n - 1 + \theta)Q_n(s) = 
 \theta Q_n(s - 1) + (n - 1)Q_{n - 1}(s),\eqno(3.3)$$
$s = 0,1, \ldots $ and $Q_1(s) = \delta_{s,0}~.$\\
Together (3.2) and (3.3) allow straightforward computation of 
$\mu_n(s;j)$ . Let
$$G_n(u,v) = \sum_{s = 0}^\infty \sum_{j = 0}^s
u^jv^s\mu_n(s;j)~,$$
then from (3.2), for $H_n(u)$ defined in (1.6),
$$(n(n - 1 + \theta) - (n - 1)\theta v)G_n(u,v)
 = n(n - 1)G_{n - 1}(u,v) + \theta uvH_n(vu)~.$$
The solution to this is \\
$$G_n(u,v) = 
\prod_{k = 2}^n
\frac{k(k - 1)}{k(k - 1 + \theta ) - (k - 1)\theta v}$$
$$~~~~~~~~~~~~~~~ +~  \theta vu \sum_{r = 2}^n
\frac{H_r(vu)}{r(r - 1)}
\prod_{k = r}^n\frac{k(k - 1)}{k(k - 1 + \theta) - (k - 1)\theta v}~
.\eqno(3.4)$$
A marginal generating function is $G_n(1,v) = H_n(v)$.

    Let $\mu_n(j)$ be the probability that a randomly chosen gene
from a sample of $n$ genes has the same allele type as the $jth$ node of
the sample's tree $T$,
with the interpretation that $\mu_n(j) = 0$ if $T$ has less than $j$ nodes.
Summing over $s \geq j$ in (3.2), 
$$(n(n - 1) + \theta )\mu_n(j) = 
n(n - 1)\mu_{n - 1}(j) + \theta Q_n(j - 1)~,$$
$n = 1,2 \ldots$ and $\mu_1(j) = \delta_{j,0}~.$

Placing $v = 1$ in (3.4) provides a generating function for $\mu_n(j)$
 and an explicit solution is
$$\mu_n(j) = \delta_{j,0}\prod_{k = 2}^n\frac{k(k - 1)}{k(k - 1) + \theta } 
+ (1 - \delta_{j,0})\sum_{r = 2}^n\frac{Q_r(j - 1)}{r(r - 1)}
\prod_{k = r}^n\frac{k(k - 1)}{k(k - 1) + \theta }.\eqno(3.5)$$
Of course $\sum_{j = 0}^\infty \mu_n(j) = 1$. Let
$\mu (j) = \lim_{n \rightarrow \infty}\mu_n(j)$. Then $\mu (j)$
has a similar form to (3.5) with $n$ replaced by $\infty$.
(The products in (3.5) are easily seen to converge.)
A particular result is
$$\mu (0) = \prod_{k = 2}^\infty\frac{k(k - 1)}{k(k - 1) + \theta} = 
\pi \theta sec\left (\frac{\pi }{2}\sqrt{1 - 4\theta } \right )~,$$
derived by Beder (1988).
One might intrepret $\{\mu (j);j = 0,1,\ldots \}$ as an expected partition of
the entire population into ancestral types. Another interpretation is
a limiting partition of the relative frequencies of types when a tree is
grown forever using the simulation (1),(2) not stopping as in (3).

    Since all ancestral types may not be represented in the population the
expected partition is different from an expected partition of those {\em in}
the population according to relative age, where the expected frequency of
the $jth$ oldest is
$$\frac{1}{1 + \theta }\left (\frac{\theta }{1 + \theta}\right )^{j - 1},
~j = 1,2, \ldots~.$$
This expected frequency is easily obtained from the age-ordered representation
of the allele frequencies in the infinitely-many-alleles model.
Let $\{Z_i,i = 1,2,\ldots \}$ be an {\em i.i.d.} sequence of random
variables with density
	$$\theta (1 - z)^{\theta - 1}~,~0 < z < 1.$$
Then the age-ordered frequencies are distributed as
$Z_1, Z_2(1 - Z_1), Z_3(1 - Z_1)(1 - Z_2), \ldots$
(Donnelly and Tavar\'{e} (1987)).

    Table 2 shows $\{\mu (j)~,~j = 0,1,\ldots ,10\}$ for illustrative values of 
$\theta.$ For $\theta$ not too large, alleles from early mutations 
take up most of the population frequency.
\vspace{0.5cm}
\begin{Large}
\begin{center}
{\bf Table 2}
\end{center}
\end{Large}
\vspace{0.5cm}
\begin{large}
\begin{center}
{\bf Relative frequencies of node types.}
\end{center}
\end{large}
\begin{tabular}{rllllll}
& & & & $\theta$ \\
Node   & 0.1 & 0.5 & 1.0 & 1.5 & 2.0 & 5.0 \\ \cline{2-7}
0 &0.9061 &0.6261 &0.4119 &0.2809 &0.1970 &0.0334 \\
1 &0.0802 &0.1886 &0.1764 &0.1396 &0.1063 &0.0213 \\
2 &0.0119 &0.0996 &0.1364 &0.1274 &0.1065 &0.0257 \\
3 &0.0016 &0.0481 &0.0981 &0.1092 &0.1009 &0.0299 \\
4 &0.0002 &0.0218 &0.0667 &0.0891 &0.0914 &0.0337 \\
5 &       &0.0094 &0.0434 &0.0699 &0.0799 &0.0370 \\
6 &       &0.0039 &0.0273 &0.0531 &0.0678 &0.0397 \\
7 &       &0.0016 &0.0166 &0.0393 &0.0561 &0.0418 \\
8 &       &0.0006 &0.0099 &0.0284 &0.0454 &0.0433 \\
9 &       &0.0002 &0.0058 &0.0201 &0.0361 &0.0441 \\
10 &      &0.0001 &0.0033 &0.0140 &0.0282 &0.0444 \\
$>$ 10 &    &       &0.0042 &0.0290 &0.0844 &0.6057
\end{tabular}
\vspace{0.5cm}

    {\em Corollary}. Let $T$ be a tree represented by $n$ age-labeled
sequences and\\ 
$q_n(r;j)
               = P$(\{The allele of the $jth$ node of $T$
has multiplicity $r$ in the sample.\})\\
Then \hfill \\
$(n(n-1)+r\theta )q_n(r;j) = n(r - 1)q_{n-1}(r-1;j) 
~+~n(n - 1 - r)q_{n-1}(r;j)
\hfill $
$$~+~(r + 1)\theta q_n(r+1;j)~+~
\delta_{r,1} n\theta Q_n(j - 1)~,\eqno(3.6)$$
$n = 2,3,\ldots~,~j = 0,1,\ldots~,~r = n,\ldots ,0$~.
The boundary condition
is $q_1(r;j) = \delta_{r,1} \delta_{j,0}$ .
Interpret $q_n(r;j)$ as $0$  $r > n$ or $r < 0$.
A particular case is 
$$q_n(n;j) = \delta_{j,0}\prod_{k = 2}^n
\frac{k-1}{k - 1 + \theta }~.$$
\vspace{0.5cm}

    {\em Proof.} Sum (3.1) over $s \geq j$ .
\vspace{0.5cm}

    {\em Corollary}.  Let $T$ be a tree represented by $n$ 
sequences and let $q_n(r) = q_n(r;0);$ 
\begin{center}
{\em i.e.} $q_n(r) = 
P$(The allele of the common ancestor has multiplicity $r$ in the sample).
\end{center}
Then \hfill \\
$$(n(n-1) + r\theta )q_n(r) = n(r - 1)q_{n-1}(r-1) 
+ n(n - 1 - r)q_{n-1}(r)
~+~(r + 1)\theta q_n(r+1)\eqno(3.7)$$ 
$n = 2,3,\ldots~,~r = n,\ldots ,0$.
 The boundary condition is
 $q_1(r) = \delta_{r,1}$ . A particular case is
$$q_n(n) = \prod_{k = 2}^n
\frac{k-1}{k - 1 + \theta }~.$$
\vspace{0.5cm}

    {\em Proof.} Let $j = 0$ in (3.6). Note that the last term vanishes.
\vspace{0.5cm}
    
    The distribution $\{q_n(r)\}$ is of particular interest.
It is easy to calculate on a computer from (3.7).
From Griffiths (1986) the mean is 
$$n\prod_{j = 2}^n\frac{j(j - 1)}{j(j - 1) + \theta }~.$$

\section{Computer implementation}

    The recursive equations (1.4) and (2.2) were implemented in the
programming language C on 8086 and Vax computers, allowing numerical
calculation of $p^*(T,{\bf  n})$ and $p^*_a(T,{\bf  n}).$

    Suppose we wish to find the probability of a tree $T$. Label the nodes
of $T$ by integers, the root being 0. All trees appearing in 
the recursion for the probability are subtrees of $T$ obtained by travelling
up the paths from the leaves to the root. Suppose the nodes are
structures containing enough information so that specifying the leaf
nodes of a subtree determines that subtree. In the author's implementation,
data structures are\\
\begin{verbatim}

typedef struct {
    int ancestor;
    int *sibs;            /* pointer to siblist */
    int sibnumber;
    int scale;
} NODE;

typedef struct {
    int *leaves;          /* pointer to leaflist */
    int *multiplicity;    /* pointer to list of multiplicities */
} TREE;
\end{verbatim}

The function implementing the recursive
equations (1.4) and (2.1)
is {\sf double P(TREE *tree,double theta,int ageflag) } returning 
$p_a(T,{\bf n})$ or $p(T,{\bf n})$ according  to whether {\sf ageflag}
is 1 or 0.
{\sf P()} is written transparently to the detail of the node data structure,  
assuming only that the tree nodes are {\sf int } labels, with the root $0$.

It is possible to have either a completely recursive implementation
or use a table lookup scheme by defining {\sf STORE} 0 or 1 for conditional
compilation.
A completely recursive scheme is sufficient for a small sample size, but is
generally inefficient, as the probability of the same 
subtree may be evaluated many times. The lookup scheme is to check whether
the probability of a subtree has been calculated; if so get it from a store,
if not calculate the probability and put it in the store. An index scheme
is needed for the subtrees of the original tree which appear in the recursion.
The author settled for a quick table lookup scheme which is reasonably
memory efficient, rather than a complex index scheme using minimal memory.
Each subtree has a unique index into an array of
double pointers {\sf pstore[]}. If there are $q$ leaves of the subtree, with
sibnumbers $m_1, \ldots ,m_q$ , in the original tree, then {\sf pstore[index]}
points to a linear ordering of a $q$-dimensional array of these dimensions. 
If the multiplicities
of the sequences of the required tree
are $n_1, \ldots ,n_q$ then the probability for the subtree 
is stored in position {\sf pstore[index][entry]}, where {\sf entry} 
is the offset
position of $(n_1,\ldots ,n_q)$.
For convenience the original leaf nodes are considered as having 
sibnumbers equal to the multiplicity of the sequences.

Calculation of the index for a subtree is done in the following way.
Let ${\bf x}_1, \ldots ,{\bf x}_d$ be paths from the leaves to the root of the
original tree with coordinates decreasing within sequences. 
Arrange the sequences so that if for any
$1 \leq i,j \leq d,~{\bf x}_i \subset {\bf x}_j$, then $i < j$. Partition the 
nodes, apart from the root, into disjoint sequences
$${\bf y}_j = \{(x_{j,0}, \ldots ,x_{j, \ell});
~x_{j,\ell } \not = 0,~~x_{j, \ell} \not\in \bigcup_{1 \leq i < j}{\bf x}_i,
~x_{j,l+1} \in \bigcup_{1 \leq i < j}{\bf x}_i\},~j = 1,\ldots ,d~.$$
That is, ${\bf y}_j$ is the portion of the path ${\bf  x}_j$
from the leaf to the root which does not include nodes in the paths
${\bf x}_1, \ldots ,{\bf x}_{j - 1}$. Let $N_0 = 1$, $N_j = \prod_{1 \leq i < j}
(|{\bf  y}_i| + 1), j = 1, \ldots d - 1$, then assign to a node $y_{j,k}$ an
integer scale value, {\sf NODE.scale} of 
$$(|{\bf  y}_j| - 1 - k)N_{j-1},~k = 0, \ldots ,|{\bf y}_j| - 1,
~j = 1, \ldots , d,$$ 
where $|{\bf  y}|$ denotes the number of elements of ${\bf  y}.$ 
The root has scale $0$.
The index of a tree is calculated by adding the scale values of the maximal
leafnodes within the partition blocks ${\bf y}_1, \ldots , {\bf y}_d$.
With this scheme
all the pointers in {\sf pstore[]} may not be used, but usually its
dimension $\prod_{j = 1}^d(|{\bf y}_j| + 1) + 1$ is not large. 
All of {\sf pstore[]}
is needed if the sequences ${\bf x}_1, \ldots ,{\bf x}_d$  are disjoint apart
from the root.
Memory is dynamically allocated
to {\sf pstore[index]} when a skeleton subtree with {\sf index} is encountered.

    To calculate the combinatorial coefficient $a(T,{\bf  n})$
first arrange the sequences ${\bf x}_1, \ldots ,{\bf x}_d$ 
so the multiplicities 
$n_1, \ldots ,n_d$ are non-decreasing. Denote $\alpha (j) = 
|\{i;n_i = j, i = 1, \ldots ,d\}|.$ Let $u_1, \ldots ,u_s$ be the distinct
elements of ${\bf x}_1, \ldots ,{\bf x}_d$ arranged in an arbitary order and
define the $s\times d$ incidence matrix $S$ by $S_{i,j} = I_{{\bf x}_j(u_i)}$,
where $I_{\bf  x}$ is the indicator function of ${\bf  x}$.
Let $Q_1, \ldots ,Q_N ,~N = \alpha (1)! \ldots \alpha (n)!$ be the
$d\times d$ permutation matrices formed by permuting $1, \ldots ,d$ within
consecutive blocks of length $\alpha (1), \ldots ,\alpha (n)$.
Then
$a(T,{\bf  n}) = |\{i~;~\exists{\rm ~a}~s\times s~{\rm permutation~matrix~}P
~{\rm such~that}~PSQ_i = S, i = 1, \ldots , N\}|.$
The author
used an algorithm from Neijenhuis and Wilf (1978) to run through permutations.

    An example is a tree constructed from sequences
and respective multiplicities
(5,0),2; (6,1,0),1; (4,1,0),1; (3,2,1,0),3.
(The reader is urged to sketch the tree!) 
The partition to calculate the scale values is
${\bf y}_1 = (5),~{\bf y}_2 = (6,1),~{\bf y}_3 = (4),~{\bf y}_4 = (3,2)$ ,
and ${\bf N} = (1,2,6,12)$ . This produces respective scale values 
(0,2,12,24,6,1,4) for the seven nodes.
The multinomial coefficient in $p^*(T,{\bf n})$
and $p_a^*(T,{\bf n})$ is 420 and $a(T,{\bf n})$ = 2, because of the
similar sequences ${\bf x}_3$ and ${\bf x}_4$ .
If a store is used in the example {\sf P()} is called 212 times, 
compared to 19,086 times for a completely recursive scheme.

    Software with convenient input and output facility, 
error checking, debugging facilities,
automatic calculation
of {\sf NODE.scale} values from an input tree, and an option of
calculating $p^*(T,{\bf n})$ , $p^*_a(T,{\bf n})$ 
is available on request from the author in source code and executable
form on a 8086/8087 family computer.
\newpage
\section{References}
{\sc Beder, B.} Allelic frequencies given the sample's common ancestral type.
{\em Theor. Pop. Biol.} {\bf 33}, 126--137. (1988).\\
{\sc Donnelly, P. and Tavar\'{e}, S.} The ages of alleles and a coalescent.
{\em Adv. Appl. Prob.} {\bf 18}, 1--19. (1986). \\
{\sc Donnelly, P. and Tavar\'{e}, S.} The population genealogy of the
infinitely-many neutral alleles model. {\em J. Math. Biol.} {\bf 25},
381--391. (1987). \\
{\sc Ethier, S. N.} The infinitely-many-neutral-alleles 
diffusion model with ages.
To appear in {\em J. Appl. Prob.} (1989).\\
{\sc Ethier, S.N. and Griffiths, R. C.} The infinitely-many-sites model 
as a measure-valued diffusion. {\em Ann. Prob.} {\bf 15}, 515--545. (1987). \\
{\sc Ewens, W. J.} The sampling theory of selectively neutral alleles.
{\em Theor. Pop. Biol.} {\bf 3}, 87--112. (1972). \\
{\sc Griffiths, R. C.} Family trees and DNA sequences.
{\em Proceedings of the Pacific Statistical Congress}, Francis, Manly,
Lam (Editors). {\em Elsevier Science Publishers.} (1986). \\
{\sc Griffiths, R. C.} Counting genealogical trees. {\em J. Math. Biol.}
{\bf 25}, 423--431. (1987).\\
{\sc Hoppe, F. M.} Polya-like urns and the Ewens sampling formula.
{\em J. Math. Biol.} {\bf 20}, 91--94. (1984). \\
{\sc Hoppe, F. M.} The sampling theory of neutral alleles and an urn model in
population genetics. {\em J. Math. Biol.} {\bf 25}, 123--157. (1987).\\
{\sc Kingman, J. F. C.} The coalescent. {\em Stoch. Proc. Appl.} {\bf 13},
235--248. (1982). \\
{\sc Nijenhuis, A., and Wilf, H. S.} Combinatorial Algorithms, 2nd edn.
{\em New York, Academic Press.} (1978).\\
{\sc Tavar\'{e}, S.} Line-of-descent and genealogical processes, and their
applications in population genetics models. {\em Theor. Pop. Biol.}
{\bf 26}, 119--164. (1984). \\
{\sc Tavar\'{e}, S.} The birth process with immigration, and the genealogical
structure of large populations. {\em J. Math. Biol.}{\bf 25}, 161-171.
 (1987). \\
{\sc Watterson, G. A.} On the number of segregating sites in genetical models
without recombination. {\em Theor. Pop. Biol.} {\bf 7}, 256--276. (1975).\\

\end{document}
